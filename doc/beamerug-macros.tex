% !TeX root = beameruserguide.tex
% Copyright 2003--2007 by Till Tantau
% Copyright 2010 by Vedran Mileti\'c
% Copyright 2013,2015 by Vedran Mileti\'c, Joseph Wright
% Copyright 2017,2018 by Louis Stuart, Joseph Wright
%
% This file may be distributed and/or modified
%
% 1. under the LaTeX Project Public License and/or
% 2. under the GNU Free Documentation License.
%
% See the file LICENSE.md for more details.

\def\beamer{\InlineClass{beamer-name}{\textsc{beamer}}}
\def\pdf{\InlineClass{pgf-name}{\textsc{pdf}}}
\def\pgfname{\InlineClass{pgf-name}{\textsc{pgf}}}
\def\translatorname{\textsc{translator}}
\def\pstricks{\textsc{pstricks}}
\def\prosper{\textsc{prosper}}
\def\seminar{\textsc{seminar}}
\def\texpower{\textsc{texpower}}
\def\foils{\textsc{foils}}

\begin{warpHTML}
\makeatletter
\renewcommand{\meta}[1]{%
  \InlineClass{meta}{\InlineClass{angle}{\HTMLentity{langle}}%
     \ifmmode \expandafter \nfss@text \fi
     {%
      \edef\meta@hyphen@restore
        {\hyphenchar\the\font\the\hyphenchar\font}%
      \hyphenchar\font\m@ne
      \language\l@nohyphenation
      \textit{#1}\/%
      \meta@hyphen@restore
     }\InlineClass{angle}{\HTMLentity{rangle}}}%
}
\makeatother
\end{warpHTML}

{
  \makeatletter
  \global\let\myempty=\@empty
  \global\let\mygobble=\@gobble
  \makeatother
  \gdef\getridofats#1@#2\relax{%
    \def\getridtest{#2}%
    \ifx\getridtest\myempty%
      \expandafter\def\expandafter\strippedat\expandafter{\strippedat#1}
    \else%
      \expandafter\def\expandafter\strippedat\expandafter{\strippedat#1\protect\printanat}
      \getridofats#2\relax%
    \fi%
  }
  \gdef\removeats#1{%
    \let\strippedat\myempty%
    \edef\strippedtext{\stripcommand#1}%
    \expandafter\getridofats\strippedtext @\relax%
  }
  \makeatletter
  \gdef\stripcommand#1{\expandafter\@gobble\string#1}
}
\providecommand\href[2]{\texttt{#1}}

\def\printanat{\char`\@}

\def\declare#1{{\InlineClass{declare}{\color{red!75!black}#1}}}
%\def\declare{\afterassignment\translatormanualdeclare\let\next=}
%\def\translatormanualdeclare{\ifx\next\bgroup\bgroup\color{red!75!black}\else{\color{red!75!black}\next}\fi}

\def\command#1{\BlockClass{command}\list{}{\leftmargin=2em\itemindent-\leftmargin\def\makelabel##1{\hss##1}}%
  \item\BlockClass{command-headline}\extractcommand#1@\par\topsep=0pt\endBlockClass}
\def\endcommand{\endlist\endBlockClass}
\def\extractcommand#1#2@{\strut\declare{\texttt{\string#1}}#2%
  \index{\stripcommand#1@\protect\myprintcommand{\stripcommand#1}}}

%\let\textoken=\command
%\let\endtextoken=\endcommand

\def\myprintcommand#1{\texttt{\char`\\#1}}

\def\example{\par\smallskip\noindent\textit{Example: }}
\def\themeauthor{\par\smallskip\noindent\textit{Theme author: }}


\def\environment#1{\BlockClass{environment}\list{}{\leftmargin=2em\itemindent-\leftmargin\def\makelabel##1{\hss##1}}%
\extractenvironement#1@\par\topsep=0pt}
\def\endenvironment{\endlist\endBlockClass}
\def\extractenvironement#1#2@{%
\item{\InlineClass{begin-environment}{\texttt{\char`\\begin\char`\{\declare{#1}\char`\}}#2}}%
  {\itemsep=0pt\parskip=0pt\item{\meta{environment contents}}%
  \item{\ttfamily\char`\\end\char`\{\declare{#1}\char`\}}}}%
  % \index{#1@\protect\texttt{#1} environment}%
  % \index{Environments!#1@\protect\texttt{#1}}}


\def\classoption#1{\BlockClass{classoption}\list{}{\leftmargin=2em\itemindent-\leftmargin\def\makelabel##1{\hss##1}}%
\item{\BlockClass{classoption-headline}{{\ttfamily\char`\\documentclass[\declare{#1}]\char`\{beamer\char`\}}}\endBlockClass}
  \index{#1@\protect\texttt{#1} class option}%
  \index{Class options for \textsc{beamer}!#1@\protect\texttt{#1}}%
  \par\topsep=0pt}
\def\endclassoption{\endlist\endBlockClass}


\newcommand\beameroption[2]{\BlockClass{beameroption}\BlockClass{beameroption-headline}\list{}{\leftmargin=2em\itemindent-\leftmargin\def\makelabel##1{\hss##1}}%
  \item{{\ttfamily\char`\\setbeameroption\char`\{\declare{#1}{\normalfont\opt{#2}}\char`\}}}
  \index{#1@\protect\texttt{#1} beamer option}%
  \index{Beamer options!#1@\protect\texttt{#1}}%
  \par\topsep=0pt\endBlockClass}
\def\endbeameroption{\endlist\endBlockClass}


\def\smallpackage{\vbox\bgroup\package}
\def\endsmallpackage{\egroup\endpackage}

\def\package#1{\BlockClass{package}\list{}{\leftmargin=2em\itemindent-\leftmargin\def\makelabel##1{\hss##1}}%
  \extracttheme#1@usepackage@package@Packages@\par\topsep=0pt}
\def\endpackage{\endlist\endBlockClass}
%\def\extracttheme#1#2@{%
%\item{{{\ttfamily\char`\\usepackage}#2{\ttfamily\char`\{\declare{#1}\char`\}}}}}

\def\theme#1#2#3#4{\list{}{\leftmargin=2em\itemindent-\leftmargin\def\makelabel##1{\hss##1}}%
\extracttheme#2@#1@#3@#4@\par\topsep=0pt}
\def\endtheme{\endlist}
\def\extracttheme#1#2@#3@#4@#5@{%
  \item{\InlineClass{extracttheme}{{{\ttfamily\char`\\#3}#2{\ttfamily\char`\{\declare{#1}\char`\}}}}}%
  \index{#1@\protect\texttt{#1} #4}%
  \index{#5!#1@\protect\texttt{#1}}
}

\def\class#1{\BlockClass{class}\list{}{\leftmargin=2em\itemindent-\leftmargin\def\makelabel##1{\hss##1}}%
\extractclass#1@\par\topsep=0pt}
\def\endclass{\endlist\endBlockClass}
\def\extractclass#1#2@{%
\item{\InlineClass{extractclass}{{{\ttfamily\char`\\documentclass}#2{\ttfamily\char`\{\declare{#1}\char`\}}}}}%
  \index{#1@\protect\texttt{#1} class}%
  \index{Classes!#1@\protect\texttt{#1}}}

\def\typesetsol#1{\texttt{\def\_{\char`\_}#1}}

\def\solution#1{\BlockClass{solution}\BlockClass{solution-headline}\list{}{\leftmargin=2em\itemindent-\leftmargin\def\makelabel##1{\hss##1}}%
\item \textbf{Solution Template }\declare{\typesetsol{#1}}\par\topsep=0pt%
  \index{#1@\protect\typesetsol{#1} solution}%
  \index{Solutions!#1@\protect\typesetsol{#1}}\endBlockClass}
\def\endsolution{\endlist\endBlockClass}

\def\template#1{\list{}{\leftmargin=2em\itemindent-\leftmargin\def\makelabel##1{\hss##1}}%
\item {\ttfamily\char`\\setbeamertemplate\char`\{\declare{#1}\char`\}}\oarg{options}\opt{\meta{args}}\par\topsep=0pt}
\def\endtemplate{\endlist}
\newenvironment{template*}[1]{\list{}{\leftmargin=2em\itemindent-\leftmargin\def\makelabel##1{\hss##1}}%
\item \leavevmode\llap{\color{blue}\vtop
    to0pt{\llap{\textsc{appear-\!}}\vskip-3pt\llap{\textsc{ance}}\vss}\ \ }{\ttfamily\char`\\setbeamertemplate\char`\{\declare{#1}\char`\}}\oarg{options}\opt{\meta{args}}\par\topsep=0pt}
{\endlist}

\newenvironment{element}[4]{\BlockClass{element}\list{}{\leftmargin=2em\itemindent-\leftmargin\def\makelabel##1{\hss##1}}%
\item \BlockClass{element-headline}{\textbf{\ifx#2\semiyes Parent Beamer-Template\else%
    Beamer\applier#2{-Template}\applier#3{\applier#2{/}-Color}\applier#4{\ifx#2\yes/\else\ifx#3\yes/\fi\fi
      -Font}\fi}
    {\ttfamily\declare{#1}}}\par\topsep=0pt\endBlockClass
  \edef\parameters{%
    \ifx#2\semiyes parent template\else%
    \applier#2{template}\applier#3{\applier#2{/}color}\applier#4{\ifx#2\yes/\else\ifx#3\yes/\fi\fi font}\fi}
  \index{#1@\protect\texttt{#1} \parameters}%
  \applier#2{\index{Beamer templates!#1@\protect\texttt{#1}}}%
  \applier#3{\index{Beamer colors!#1@\protect\texttt{#1}}}%
  \applier#4{\index{Beamer fonts!#1@\protect\texttt{#1}}}%
}
{\endlist\endBlockClass}

\def\applier#1#2{\ifx#1\yes#2\fi}

\def\templateoptions{\par
  The following template options are predefined:
  \begin{itemize}}
\def\endtemplateoptions{\end{itemize}}

\def\itemoption#1#2{\item\InlineClass{itemoption}{{\texttt{[\declare{#1}]}#2}}}

%\def\itemoption#1{\item \declare{\texttt{#1}}%
%  \indexoption{#1}%
%}

%\def\indexoption#1{%
%  \index{#1@\protect\texttt{#1} option}%
%  \index{Options!#1@\protect\texttt{#1}}%
%}

\def\yes{\hbox to .6cm{\ding{51}\hfil}}
\def\semiyes{\hbox to .6cm{(\ding{51})\hfil}}
\def\no{\hbox to .6cm{\ding{55}\hfil}}

\def\choosecol#1{}%\ifx#1\yes\color{green!50!black}\else\color{red!50!black}\fi}

\def\templatefontcolor#1#2#3#4{%
  \item\declare{\texttt{#1}}\hfill%
  {\choosecol#2Template #2} {\choosecol#3Color #3} {\choosecol#4Font #4}\par}

\def\fontparents#1{Font parents: \texttt{#1}\par}
\def\colorparents#1{Color parents: \texttt{#1}\par}
\def\colorfontparents#1{Color/font parents: \texttt{#1}\par}

\def\templateinserts{\begin{itemize}}
\def\endtemplateinserts{\end{itemize}}

\def\iteminsert#1{\item {\texttt{\declare{\string#1}}}%
  \index{Inserts!\stripcommand#1@\protect\myprintcommand{\stripcommand#1}}}

\newcommand\opt[1]{\InlineClass{opt}{\color{black!50!green}#1}}
\renewcommand\oarg[1]{\opt{\InlineClass{oarg-left-bracket}{\ttfamily[}\meta{#1}\InlineClass{oarg-right-bracket}{\ttfamily]}}}
\newcommand\ooarg[1]{{\ttfamily[}\meta{#1}{\ttfamily]}}
\newcommand\sarg[1]{\opt{{\ttfamily\char`\<}\meta{#1}{\ttfamily\char`\>}}}
\newcommand\ssarg[1]{{\ttfamily\char`\<}\meta{#1}{\ttfamily\char`\>}}

%\def\opt{\afterassignment\translatormanualopt\let\next=}
\def\translatormanualopt{\ifx\next\bgroup\bgroup\color{black!50!green}\else{\color{black!50!green}\next}\fi}

\newcommand{\beamernote}{\warpprintonly{\par\smallskip\noindent\llap{\color{blue}\vtop to0pt{\llap{\textsc{presen-\!}}\vskip-3pt\llap{\textsc{tation}}\vss}\ \ }}%
  \warpHTMLonly{\BlockClass{beamernote}\par\smallskip\noindent\textsc{presentation}\endBlockClass}}
\newcommand{\articlenote}{\warpprintonly{\par\smallskip\noindent\llap{\color{blue}\textsc{article}\ \ }}%
  \warpHTMLonly{\BlockClass{articlenote}\par\smallskip\noindent\textsc{article}\endBlockClass}}
\newcommand{\appearancenote}{\par\smallskip\noindent\appearancenotetext}

\def\appearancenotetext{\llap{\color{blue}\vtop
    to0pt{\llap{\textsc{appear-\!}}\vskip-3pt\llap{\textsc{ance}}\vss}\ \ }}

\newcommand{\templatenote}{\par\smallskip\noindent\llap{\color{blue}\textsc{template}\ \ }}
\newcommand{\colornote}{\par\smallskip\noindent\llap{\color{blue}\textsc{color}\ \ }}
\newcommand{\fontnote}{\par\smallskip\noindent\llap{\color{blue}\textsc{font}\ \ }}

\newcommand{\genericthemeexample}[2][]{%
  \smallskip\par\noindent\BlockClass{themeexample-image}%
    \includegraphics[width=8cm,page=1]{beamerug#2-page1}\qquad\includegraphics[width=8cm,page=2]{beamerug#2-page2}%
  \endBlockClass  
  \smallskip\par}

\newenvironment{themeexample}[2][]
{\BlockClass{theme}\begin{theme}{usetheme}{{#2}#1}{presentation theme}{Presentation themes}
    \example\genericthemeexample{theme#2}
  }
{\end{theme}\endBlockClass}

\newenvironment{innerthemeexample}[2][]
{\BlockClass{innertheme}\begin{theme}{useinnertheme}{{#2}#1}{inner theme}{Inner themes}
    \example\genericthemeexample{innertheme#2}
  }
{\end{theme}\endBlockClass}

\newenvironment{outerthemeexample}[2][]
{\BlockClass{outertheme}\begin{theme}{useoutertheme}{{#2}#1}{outer theme}{Outer themes}
    \example\genericthemeexample{outertheme#2}
  }
{\end{theme}\endBlockClass}

\newenvironment{colorthemeexample}[2][]
{\BlockClass{colortheme}\begin{theme}{usecolortheme}{{#2}#1}{color theme}{Color themes}
    \example\genericthemeexample{colortheme#2}
  }
{\end{theme}\endBlockClass}

\newenvironment{fontthemeexample}[2][]
{\BlockClass{fonttheme}\begin{theme}{usefonttheme}{{#2}#1}{font theme}{Font themes}
    \example\genericthemeexample{fonttheme#2}
  }
{\end{theme}\endBlockClass}

\newenvironment{fontthemeexample*}[2][]
{\begin{theme}{usefonttheme}{{#2}#1}{font theme}{Font themes}}
{\end{theme}}

\def\partname{Part}

\colorlet{examplefill}{yellow!80!black}
\definecolor{graphicbackground}{rgb}{0.96,0.96,0.8}
\definecolor{codebackground}{rgb}{0.8,0.8,1}

\newenvironment{translatormanualentry}{\list{}{\leftmargin=2em\itemindent-\leftmargin\def\makelabel##1{\hss##1}}}{\endlist}
\newcommand\translatormanualentryheadline[1]{%
  {\itemsep=0pt\parskip=0pt\item\strut#1\par\topsep=0pt}}
\newcommand\translatormanualbody{\parskip3pt}


%\newenvironment{command}[1]{
%  \begin{translatormanualentry}
%    \extractcommand#1\@@
%    \translatormanualbody
%}
%{
%  \end{translatormanualentry}
%}

%\def\extractcommand#1#2\@@{%
%  \translatormanualentryheadline{\declare{\texttt{\string#1}}#2}%
%  \removeats{#1}%
%  \index{\strippedat @\protect\myprintcommand{\strippedat}}}


\renewenvironment{environment}[1]{
  \BlockClass{environment}%
  \begin{translatormanualentry}
    \extractenvironement#1\@@
    \translatormanualbody
}
{
  \end{translatormanualentry}\endBlockClass
}

\def\extractenvironement#1#2\@@{%
  \BlockClass{environment-headline}%
  \translatormanualentryheadline{\BlockClass{begin-environment}{\ttfamily\char`\\begin\char`\{\declare{#1}\char`\}}#2\endBlockClass}%
  \translatormanualentryheadline{{\ttfamily\ \ }\meta{environment contents}}%
  \translatormanualentryheadline{\BlockClass{end-environment}{\ttfamily\char`\\end\char`\{\declare{#1}\char`\}}\endBlockClass}%
  \endBlockClass
  \index{#1@\protect\texttt{#1} environment}%
  \index{Environments!#1@\protect\texttt{#1}}}



%\newenvironment{package}[1]{
%  \begin{translatormanualentry}
%    \translatormanualentryheadline{{\ttfamily\char`\\usepackage\opt{[\meta{options}]}\char`\{\declare{#1}\char`\}}}
%    \index{#1@\protect\texttt{#1} package}%
%    \index{Packages and files!#1@\protect\texttt{#1}}%
%    \translatormanualbody
%}
%{
%  \end{translatormanualentry}
%}



\newenvironment{filedescription}[1]{
  \begin{translatormanualentry}
    \translatormanualentryheadline{File {\ttfamily\declare{#1}}}%
    \index{#1@\protect\texttt{#1} file}%
    \index{Packages and files!#1@\protect\texttt{#1}}%
    \translatormanualbody
}
{
  \end{translatormanualentry}
}


\newenvironment{packageoption}[1]{
  \begin{translatormanualentry}
    \translatormanualentryheadline{{\ttfamily\char`\\usepackage[\declare{#1}]\char`\{translator\char`\}}}
    \index{#1@\protect\texttt{#1} package option}%
    \index{Package options for \textsc{translator}!#1@\protect\texttt{#1}}%
    \translatormanualbody
}
{
  \end{translatormanualentry}
}

\makeatletter
\def\index@prologue{\let\toclevel@section\toclevel@part\section*{Index}\addcontentsline{toc}{section}{Index}
  This index only contains automatically generated entries, sorry. A good
  index should also contain carefully selected keywords.
  \bigskip
}
\c@IndexColumns=2
  \def\theindex{\@restonecoltrue
    \columnseprule \z@  \columnsep 35\p@
    \twocolumn[\index@prologue]%
       \parindent -30pt
       \columnsep 15pt
       \parskip 0pt plus 1pt
       \leftskip 30pt
       \rightskip 0pt plus 2cm
       \small
       \def\@idxitem{\par}%
    \let\item\@idxitem \ignorespaces}
  \def\endtheindex{\onecolumn}
\def\noindexing{\let\index=\@gobble}

\makeatother
